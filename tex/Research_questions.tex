\section{Research Question and Approach}

\subsection{Climate Trends}

We investigate how the temperature in Sweden has changed over time using historical data from 1850 to 2024. The analysis focuses on long-term trends in maximum, minimum, and mean temperatures, both for a single station in Lund and for Sweden as a whole. The goal is to determine whether there is a measurable long-term change and to quantify the trend that describes the evolution of the Swedish climate.

The temperature variation over time is approximated as a linear trend, and a linear fit is applied to estimate the rate of change. This fitted model also provides a simple way to project possible future temperature patterns.



\subsection{Birthday Temperature Trends}
We chose to investigate the temperature changes throughout time for specific days, spanning a period from 100 and 200 years ago up to the present. The goal was to find if the temperature of each day changed by a similar amount over the years and, if not, to identify when during the year the temperature changes the most. \\

To consider the different climates in the south and north of Sweden, we chose to study the temperatures in Lund and Luleå. 

\subsection{Solar Activity Period Estimation}
We use the temperature data across Sweden to see if the solar activity cycle period can be estimated. We adjust the temperature, normalize it and average it to get a timeseries of adjusted data.\\

We expect this time series to have a peak frequency in line with the known solar activity frequency.