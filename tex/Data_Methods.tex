\section{Data and Methods}

The data used in this project come from the Swedish Meteorological and Hydrological Institute (SMHI) and are provided as open datasets in CSV format, covering the years 1850–2024. Each file contains daily temperature measurements from several Swedish weather stations, along with metadata such as measurement quality indicators.

In the preprocessing stage, all data points flagged as low quality were removed using the provided quality information. Unnecessary metadata fields were also discarded to simplify the dataset. The cleaned data were then sorted into folders corresponding to the different parts of the analysis. The C++ analysis, discussed below, was performed on these filtered CSV files.

For efficient handling and statistical analysis, the processed CSV files were converted into \texttt{.root} files, which were then used for plotting and fitting in ROOT.

The entire workflow, from data filtering to final visualization, is automated using a set of Bash scripts.

\subsection{Climate Trends}
The climate trend analysis was performed by calculating the annual maximum, minimum, and mean temperatures for each weather station. The same calculations were then repeated using data from all stations combined to represent Sweden as a whole. The processed data was saved in CSV format and imported into ROOT, where the temperature trends were plotted and fitted with linear functions.

\subsection{Birthday Temperature Trends}
The analysis of the temperature on our birthdays began with filtering the data to keep only the correct days each year. To be able to compare the temperatures on our birthdays accurately, the data was filtered again to only include the temperature between times 10:00 and 15:00. This resulted in some days having more data points than others and an average was taken of all the temperatures for each day. Having the resulting data sorted in time, the next step was to group the points of each day to make it easier to plot them separately. The average temperature of each day was then plotted over time to compare the temperatures between the days and see how the average temperature on each day changes over time. 


\subsection{Solar Activity Period Estimation}
The analysis for the solar activity period estimation began by filtering the temperatures to be some time around midday (11:00 to 15:00 UTC), this was done in preparation for the next step. The next step was to adjust the temperatures based on what the expected solar output would be given the time, date and location. Filtering to times around midday allows for mitigation of some effects that would be caused by the atmosphere, as well as avoid times before sunrise and after sunset where the solar output would have been 0. With these adjusted temperatures, which essentially assume the sun is directly overhead, the next steps can be taken. This is described in the equations below.\\

The Equation of Time (EoT) quantifies the difference between solar time and mean clock time due to Earth's axial tilt and orbital eccentricity.  
It is given in minutes as:
\begin{align}
    B &= \frac{2\pi (J - 81)}{364} \label{eq:B} \\
    E_{\mathrm{oT}} &= 9.87 \sin(2B) - 7.53 \cos(B) - 1.5 \sin(B) \label{eq:EoT}
\end{align}
where \( J \) is the day of the year (\( J=1 \) for January 1). Where 81 is the approximate day of the year of the march equinox, and 364 is used instead of 365 to symmetrize the annual period.  \( B \) is the normalized angular position of earth in its orbit. The other numbers in the equation are derived from empirical data, on earths eccentricity as well as the behavior of the sun.\\

The solar declination angle \( \delta \) (in radians) is the angular position of the Sun north or south of the celestial equator:
\begin{equation}
    \delta = 23.45^\circ \times \frac{2\pi}{180^\circ} \times 
    \sin\!\left( B \right)
\end{equation}
where \(23.45^\circ\) is earths axial tilt.\\

Because Earth's orbit is slightly elliptical, the distance to the Sun varies through the year.  
The eccentricity correction factor \( E_0 \) adjusts the solar constant:
\begin{equation}
    E_0 = 1 + 0.033 \cos\!\left( \frac{2\pi J}{365} \right)
\end{equation}
where 0.033 is an estimate of earths eccentricity.\\

The local solar time (LST) in hours is given by
\begin{equation}
    \mathrm{LST} = h_{\mathrm{UTC}} + \frac{4\lambda + E_{\mathrm{oT}}}{60}
\end{equation}
where \(h_{\mathrm{UTC}}\) is the time given in the dataset, and \(\lambda\) is the longitude also from the dataset. The 4 comes from the fact that 4 minutes occur for every degree earth rotates(\(4 \frac{min}{^\circ}\), and the 60 converts that to hours.\\

The solar hour angle (in radians) is then
\begin{equation}
    H = 15^\circ  (\mathrm{LST} - 12)  \frac{2\pi}{180^\circ}
\end{equation}
where the \(15^\circ\) is the amount earth rotates per hour, and 12 is local noon on the local meridian.\\

The cosine of the solar zenith angle, \( \mu_0 = \cos\theta_z \), depends on the observer's latitude \( \varphi \), the solar declination \( \delta \), and the solar hour angle \( H \):
\begin{equation}
    \mu_0 = \sin\varphi \, \sin\delta + \cos\varphi \, \cos\delta \, \cos H
\end{equation}

The top-of-atmosphere (TOA) horizontal irradiance \( G_{0h} \) at UTC hour \(h_{\mathrm{UTC}}\) is
\begin{equation}
    G_{0h}(J, h_{\mathrm{UTC}}, \lambda, \varphi) =
    \begin{cases}
        I_{\mathrm{sc}} \, E_0 \, \mu_0, & \mu_0 > 0 \\
        0, & \text{otherwise.}
    \end{cases}
\end{equation}
where \(I_{\mathrm{sc}} \approx 1367\frac{W}{m^2}\) is the standard output of the sun at earths distance. This accounts for solar declination, local hour angle, and orbital effects.\\  
When \( \mu_0 \le 0 \), the Sun is below the horizon and no irradiance is received.\\

For a given year (with \( D = 365 \) or \( 366 \) days), the mean TOA irradiance for a fixed UTC hour is
\begin{equation}
    \bar{G}_{0h}(h_{\mathrm{UTC}}, \lambda, \varphi) =
    \frac{1}{D} \sum_{J=1}^{D} G_{0h}(J, h_{\mathrm{UTC}}, \lambda, \varphi)
\end{equation}

These 2 values are then subtracted and multiplied by a correlation factor \(\beta = 0.003 \frac{^\circ C m^2}{W}\) to arrive at the amount that would be subtracted from the temperature.
\begin{equation}
    t_{adj} = t_{data} - \beta(G_{0h}-\bar{G}_{0h})
\end{equation}
The value of \(\beta\) was chosen somewhat arbitrarily but another investigation would optimize it to minimize the variance of \(t_{adj}\) to eliminate atmospheric effects as much as possible. In such a case the next step can be skipped.\\

Next the data was compiled by day of year, so that for each day of the year, there were data-points for each location and year (time could be ignored at this point). This was used to normalize the data for each day to a value between 0 and 1. This was done so that temperatures between the days of year could be compared. Now these normalized values were averaged for each month over the data, this data was plotted. Then a fast fourier transform was done on this data to see which frequencies were dominant, this was also converted into periods to make it easier to examine.
