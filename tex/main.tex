

\def\nameoflab{Temperature Trends in Sweden}
\def\coursecode{MNXB11}

\def\authorOne{Jens Hieronymus, Lukas Nord, Linn Preuss Jelvez}
\def\authorOneMail{}
\def\headerName{}

\def\supervisor{}
\def\supervisorMail{}

\documentclass[12pt,titlepage]{article}
\input{preamble.tex} 

% Use option lineno for line numbers 

\usepackage{float}
\usepackage{subcaption}
\usepackage{caption}
\usepackage{graphicx}




%\title{Temperature Trends in Sweden}

%\author[1]{Jens Hieronymus}
%\author[1]{Lukas Nord}
%\author[1]{Linn Preuss Jelvez}
%\affil[1]{Group 8}

\begin{document}
\begin{titlepage}
    \vspace*{\fill}	\newcommand{\HRule}{\rule{\linewidth}{0.5mm}}
	\center 
	\HRule\\[0.4cm]
	{\huge\bfseries \nameoflab}\\[0.4cm]
        {\Large\bfseries \coursecode}\\[0.1cm]
	\HRule\\[1.5cm]

    \large
    \textit{Author}\\
    \authorOne  \\ 

    \vspace{1cm}
     
    Lund University\\
    Department of Physics
     
    
    \bigbreak
    \vfill
    \today
\end{titlepage}

%\maketitle
% \begin{abstract}
% Please provide an abstract of no more than 300 words. Your abstract should explain the main contributions of your article, and should not contain any material that is not included in the main text. 
% \end{abstract}



\flushbottom

\thispagestyle{empty}

\section*{ABSTRACT}
This project examines temperature trends in Sweden using open data from the Swedish Meteorological and Hydrological Institute covering the years 1850 to 2024. Three main analyses were carried out to study climate change, temperature patterns on specific days, and possible links between solar activity and temperature variation. The data were cleaned, processed, and analyzed using statistical methods and Fourier transforms.

The results show a clear warming trend across Sweden of about one degree Celsius per century. In Lund, the mean annual temperature has risen slightly, while the range between maximum and minimum temperatures has widened. This suggests that summers are becoming warmer and winters colder. The study of birthday temperatures showed a similar gradual increase over time with consistent patterns across both Lund and Luleå. The analysis of solar activity did not reveal any strong periodic trend related to the known eleven-year solar cycle, indicating that such effects are not easily detectable in Swedish temperature data.

Overall, the findings confirm a steady warming of the Swedish climate consistent with global patterns. The project demonstrates how open meteorological data and simple computational tools can be used to visualize and understand changes in climate.
%\keywords{Solar Cycle, Climate, Birthdays}



\section{Introduction}

Everyone knows the weather, but few people have actually analyzed the data we publicly have access to. We decided to look at this data and see what could actually be inferred from it. We set out to do 3 investigations:
\begin{enumerate}
    \item Climate Trends
    \item Birthday Temperature Trends
    \item Solar Activity Period Estimation
\end{enumerate}

We are using the Swedish Meteorological and Hydrological Institute (SMHI), since that is the country we live in.

\subsection{Climate Trends}
We study how Sweden’s temperature has changed from 1850 to 2024 by analyzing trends in maximum, minimum, and mean temperatures for both the Lund station and the country as a whole to quantify the general climate trend.

\subsection{Birthday Temperature Trends}


{\Huge \textcolor{red}{We are born \\ and \\ We die}}\\\textcolor{red}{/Homer Simpson (perchance)}\\

We were curious to see how the temperature has historically changed over our birthdays [todo].

\subsection{Solar Activity Period Estimation}
The solar activity is usually measured by looking at the number of sunspots on the sun. Instead of this we wanted to see if this increase in activity, and thus luminosity would impact the temperature in Sweden and if this period could be inferred.

% Purpose: Briefly explain why studying temperature trends over long periods is important (e.g., climate change indicators, regional variation, etc.).

% Scope: Describe what the project investigates:

% Long-term mean temperature evolution in Sweden

% Temperature trends on specific calendar dates (e.g., birthdays)

% Possible correlations with solar events

% Data Source: SMHI open climate dataset (1850–2024).

\section{Data and Methods}

The data used in this project come from the Swedish Meteorological and Hydrological Institute (SMHI) and are provided as open datasets in CSV format, covering the years 1850–2024. Each file contains daily temperature measurements from several Swedish weather stations, along with metadata such as measurement quality indicators.

In the preprocessing stage, all data points flagged as low quality were removed using the provided quality information. Unnecessary metadata fields were also discarded to simplify the dataset. The cleaned data were then sorted into folders corresponding to the different parts of the analysis. The C++ analysis, discussed below, was performed on these filtered CSV files.

For efficient handling and statistical analysis, the processed CSV files were converted into \texttt{.root} files, which were then used for plotting and fitting in ROOT.

The entire workflow, from data filtering to final visualization, is automated using a set of Bash scripts.

\subsection{Climate Trends}
The climate trend analysis was performed by calculating the annual maximum, minimum, and mean temperatures for each weather station. The same calculations were then repeated using data from all stations combined to represent Sweden as a whole. The processed data was saved in CSV format and imported into ROOT, where the temperature trends were plotted and fitted with linear functions.

\subsection{Birthday Temperature Trends}
The analysis of the temperature on our birthdays began with filtering the data to keep only the correct days each year. To be able to compare the temperatures on our birthdays accurately, the data was filtered again to only include the temperature between times 10:00 and 15:00. This resulted in some days having more data points than others and an average was taken of all the temperatures for each day. Having the resulting data sorted in time, the next step was to group the points of each day to make it easier to plot them separately. The average temperature of each day was then plotted over time to compare the temperatures between the days and see how the average temperature on each day changes over time. 


\subsection{Solar Activity Period Estimation}
The analysis for the solar activity period estimation began by filtering the temperatures to be some time around midday (11:00 to 15:00 UTC), this was done in preparation for the next step. The next step was to adjust the temperatures based on what the expected solar output would be given the time, date and location. Filtering to times around midday allows for mitigation of some effects that would be caused by the atmosphere, as well as avoid times before sunrise and after sunset where the solar output would have been 0. With these adjusted temperatures, which essentially assume the sun is directly overhead, the next steps can be taken. Next the data was compiled by day of year, so that for each day of the year, there were data-points for each location and year (time could be ignored at this point). This was used to normalize the data for each day to a value between 0 and 1. This was done so that temperatures between the days of year could be compared. Now these normalized values were averaged for each month over the data, this data was plotted. Then a fast fourier transform was done on this data to see which frequenciers were dominant, this was also converted into periods to make it easier to examine

\section{Research Question and Approach}

\subsection{Climate Trends}

We investigate how the temperature in Sweden has changed over time using historical data from 1850 to 2024. The analysis focuses on long-term trends in maximum, minimum, and mean temperatures, both for a single station in Lund and for Sweden as a whole. The goal is to determine whether there is a measurable long-term change and to quantify the trend that describes the evolution of the Swedish climate.

The temperature variation over time is approximated as a linear trend, and a linear fit is applied to estimate the rate of change. This fitted model also provides a simple way to project possible future temperature patterns.



\subsection{Birthday Temperature Trends}
We chose to investigate the temperature changes throughout time for specific days, spanning a period from 100 and 200 years ago up to the present. The goal was to find if the temperature of each day changed by a similar amount over the years and, if not, to identify when during the year the temperature changes the most. \\

To consider the different climates in the south and north of Sweden, we chose to study the temperatures in Lund and Luleå. 

\subsection{Solar Activity Period Estimation}
We use the temperature data across Sweden to see if the solar activity cycle period can be estimated. We adjust the temperature, normalize it and average it to get a timeseries of adjusted data.\\

We expect this time series to have a peak frequency in line with the known solar activity frequency.

\newpage
\section{Results, Discussion and Conclusion}

Here we present and discuss the findings for the different research questions mentioned above. Each research question is treated separately.

% 4.1 Long-Term Mean Temperature

% Plot(s) showing mean temperature over time.

% Linear fit with slope in °C per century.

% Comment on uncertainty or variability.

% 4.2 Birthday Trends

% Temperature trends for selected dates.

% Observations on seasonal or personal date differences.

% 4.3 Solar Event Analysis

% Visuals comparing temperature and solar activity.

% Note on whether patterns are correlated or independent.
\subsection{Long-Term Temperature Trends}

To investigate the long-term changes in climate, the maximum, minimum, and mean temperatures were analyzed over the period 1850--2024. Linear fits were applied to estimate the overall rate of temperature change in both individual stations and the combined Swedish dataset.


Figure \ref{fig:lund_maxmin} and \ref{fig:lund_mean} shows the temperature trends for the Lund station. The results show an increase in maximum temperature and a decrease in minimum temperature over the studied period. This implies that while summers have become warmer, winters or night-time temperatures have become colder. Consequently, the annual temperature range has widened, suggesting stronger variability. The mean annual temperature, however, still shows a slight positive linear trend, consistent with an overall gradual warming of approximately one degree per century.

\begin{figure}[H]
    \centering
%    \begin{subfigure}[b]{0.48\textwidth}
        \centering
        \includegraphics[width=0.9\textwidth]{plots/max_min_temps/Lund_max_min_trends.pdf}
        \caption{Long-term temperature trends at the Lund station (1850--2024). 
    Linear fits are shown for maximum and minimum temperatures. The fits shows an increase in maximmum annual temperature of about four degrees per century and a decrease in minimun annual temperature of about three degrees per century.}
        \label{fig:lund_maxmin}
%    \end{subfigure}
\end{figure}
%    \hfill
\begin{figure}[H]
%    \begin{subfigure}[b]{0.48\textwidth}
        \centering
        \includegraphics[width=0.9\textwidth]{plots/mean_temps/Lund_mean_trend.pdf}
        \caption{Long-term temperature trends at the Lund station (1850--2024). 
    Linear fits are shown for mean annual temperature. The fit shows an increase in temperature of about 1 degree per century.}
        \label{fig:lund_mean}
%    \end{subfigure}
%    \caption{Long-term temperature trends at the Lund station (1850--2024). 
%    Linear fits are shown for maximum, minimum, and mean temperatures.}
%    \label{fig:lund_trends}
\end{figure}


The results from multiple stations were combined to represent the general temperature trend in Sweden, as shown in Figure \ref{fig:sweden_trend}. The combined data show a positive trend, with average temperatures increasing by about one degree per century. The consistency across different stations indicates that the observed warming is not localized, but reflects a general national trend.

\begin{figure}[H]
    \centering
    \includegraphics[width=0.9\textwidth]{plots/mean_temps/Sweden_mean_trend.pdf}
    \caption{Combined long-term temperature trend for Sweden based on multiple stations (1850--2024). The fit shows an overall warming of about one degree per century.}
    \label{fig:sweden_trend}
\end{figure}

\subsubsection{Discussion}

The temperature analysis for both the Lund station and the combined Sweden dataset shows a clear climate impact on the temperature from the mid-19th century to the present day. The linear fits for mean, maximum, and minimum temperatures all indicate slopes toward the extremes, suggesting that Sweden has experienced a gradual change in temperature over the past 170 years.

While the warming trend is consistent with global climate change observations, some local variations are visible. Lund, for example, shows slightly larger fluctuations and a stronger trend in minimum temperature, which may reflect urban heat effects or regional climate differences. The national average tends to smooth out such variations, giving a clearer picture of the overall climate trend in Sweden.

It is important to note that a linear approximation is a simplification. Temperature evolution is influenced by many factors, including natural variability, solar activity, oceanic cycles, and greenhouse gas emissions. Limited resolution and occasional data gaps, especially in older records, also introduce uncertainty in the estimated trends.

\subsubsection{Conclusion}

The analysis confirms a significant impact on the temperature trend across Sweden, with both maximum and minimum annual temperatures going toward the extremes since 1850. The results align with broader global climate observations and provide clear evidence of regional climate change effects. We can see that Sweden will experience colder winters and warmer summers, with the average temperature increasing.

For future work, more detailed modeling could include seasonal decomposition or non-linear fits to capture shorter-term oscillations and long-term variability. Combining temperature data with other meteorological parameters such as precipitation, solar radiation, or atmospheric CO$_2$ concentration would also allow for a clearer view of the driving factors behind these trends.

Overall, the study demonstrates how historical climate data and modern analysis tools like ROOT can be used to visualize and quantify long-term climate evolution in a clear and reproducible way.

\subsection{Temperature changes on specific days}
The first city to be studied was Lund and the average temperatures for each day are presented in Figure \ref{fig:lund}. 
\begin{figure}
    \centering
    \includegraphics[width=0.5\linewidth]{plots/Lundbdays.pdf}
    \caption{Caption}
    \label{fig:lund}
\end{figure}

\begin{figure}
    \centering
    \includegraphics[width=0.5\linewidth]{plots/Luleabdays.pdf}
    \caption{Caption}
    \label{fig:luleå}
\end{figure}



\subsection{Solar Activity Period Estimation}
We can see from figure \ref{fig:monthly_temp} that there is no clear trend or period in the data. It is worth pointing out that there are no values at 1 or 0 which would be expected from normalized data as the data has been averaged.

\begin{figure}[H]
    \centering
%    \begin{subfigure}[b]{0.48\textwidth}
        \centering
        \includegraphics[width=0.9\textwidth]{../plots/solar/monthly_norm_temp_timeline.png}
        \caption{This graph shows the normalized monthly temperature averaged for each month plotted in succession.}
        \label{fig:monthly_temp}
%    \end{subfigure}
\end{figure}
To be able to see what frequencies are actually relevant a fast fourier transform was done and the frequencies are plotted in \ref{fig:freq}.

\begin{figure}[H]
%    \begin{subfigure}[b]{0.48\textwidth}
        \centering
        \includegraphics[width=0.9\textwidth]{../plots/solar/solar_frequency_analysis.png}
        \caption{This graph shows the most relevant frequencies in the data from \ref{fig:monthly_temp}.}
        \label{fig:freq}
\end{figure}


To be able to more clearly see the actual periods, we inverted this graph to show information more clearly.

\begin{figure}[H]
    \centering
    \includegraphics[width=0.9\textwidth]{../plots/solar/period_analysis.png}
    \caption{periods of data in the \ref{fig:freq} graph.}
    \label{fig:period}
\end{figure}

\subsubsection{Discussion}

It is quite clear to see the data is not great. There is a lot of noise and none of the frequencies really stand out. It can quickly be looked up that the period of solar activity is around 11 years, and while \ref{fig:period} shows a peak there, the peak is pretty weak compared to everything else. This can also be seen in the \ref{fig:freq} as the frequency graph looks pretty much like a step function, with some noise. This disagrees with the original expectation that this time series would have a peak frequency in line with the known solar activity frequency.

To improve this study in the future a lot of things need to be done. Firstly data should be taken from more places around the world rather than limited to Sweden. Also meteorological effects should be taken into account as a cloudy day could mess with the data. Additionally rather than averaging every month, it may be more insightful to use a smoothing algorhithm on the data as that would remove some of the high frequency noise. Also the other sources of noise should be understood to be eliminated.

\subsubsection{Conclusion}

Data disagrees with the research question which means that either the solar cycle cannot be estimated using the temperature data, more data is needed, or data needs to be processed to a higher standard. There was a peak present at the expected 11 years, but it wasn't a major peak, indicating that it was not the most relevant in the datas frequencies.


%

\section{Discussion and Conclusion}


% Interpretation of main findings.

% How consistent are the results with expected climate trends?

% Possible sources of error (measurement gaps, station changes, etc.).

% Limitations and potential improvements (e.g., adding more stations or using homogenized data).


% Summarize main trends observed.

% Give the estimated warming rate (e.g., X °C/century).

% Mention any interesting secondary results (birthday or solar analysis).

% Reflect briefly on the learning outcomes and challenges.



%\bibliography{sample}

\end{document}
